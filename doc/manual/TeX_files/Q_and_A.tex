\chapter{Questions from end users}

\textbf{Do you have recommendations on how to best record data?} \\
I recommend stop/scan mode for most accurate mapping.
Continuous mapping is for increase the time of the survey.  \\
\textbf{How much distance can be between two consecutive start/stop acquisitions?}\\
I suggest not more than 10 meters.\\
\textbf{Do they need to overlap? To which degree?}\\
Stop/scans should be overlapped at least 50\%.\\
\textbf{Continuous scanning: can the sensor change its tilt/angle during the recording phase? 
Or does it assume being in a upright position all the time?} \\
I suggest that MANDEYE  is somehow a upright position all the time. \\
\textbf{How “fast” am I allowed to move (I actually did a rather slow walk).} \\
I was tested it up to 8 km/h\\
\textbf{Can the sensor change height while recording?}\\
Yes.\\
\textbf{Why first scan is blurry?}\\
You should follow \url{https://github.com/JanuszBedkowski/mandeye_controller/blob/main/doc/manual/manual_v0_2/mandeye_dev_manual_v0_2.pdf} - section 2.2 turn on continous scanning (MANDEYE DEV/PRO)\\
\textbf{Do you have a video of operating MANDEYE DEV?}\\
I am planning launching MANDEYE YouTube channel ASAP.\\
\textbf{Why there are 2 operations of stopping the scan?}\\
MANDEYE is working within single session scheme. It means all data will be recorded to one folder. MANDEYE will make new folder after turn of turn on procedure.\\
\textbf{The usb that came with mandeye DEV has files in it, do i need to format it?}\\
If You have MANDEYE from me (januszbedkowski@gmail.com) then Release programs and manuals are on USB. Please do not format it, just use it.\\
\textbf{How much can i scan in one session?}\\
I suggest not more than 5km. It will be much easier to work with such session. Obviously long single session can be split into sub sessions, so dont worry if You make any mistake during data collection.\\




 
